\chapter{Einleitung}
Die folgende Arbeit thematisiert die Anwendung eines \ac{ML}-Algorithmus, um einen Aktienkurs vorherzusagen. Unter der Verwendung eines geeigneten Datensatzes wird ein Modell entwickelt, welches den Aktienkurs eines Unternehmens für einen bestimmten Zeitraum vorhersagt.

\section{Motivation und Zielsetzung}\label{sec:motivation}
Der glamouröseste aller Finanzmärkte ist der Aktienmarkt, der immer wieder Bilder von Aktienhändlern, die hektisch Aktien kaufen und verkaufen und Millionen verdienen, hervorruft. Gemäß einer Statistik von statista betrug das Volumen des weltweiten Aktienhandels 2020 etwa 185,7 Billionen Dollar \parencite{Statista.22.03.2022}. Leider zeichnet die Realität der Natur des Marktes ein weniger optimistisches Bild. Der Aktienmarkt ist im Wesentlichen ein nicht lineares, nicht parametrisches System, das extrem schwer mit vernünftiger Genauigkeit zu modellieren ist \parencite[vgl.][S. 13]{Wang.2003}. Deswegen fällt es sehr schwer, den Verlauf einer Aktie vorherzusagen. Wenn der Kurs einer Aktie jedoch für einen gewissen Zeitraum vorhergesagt werden könnte, würden sich die Verluste, die Menschen an der Börse hinnehmen müssen, erheblich reduzieren.


Ziel dieser Ausarbeitung ist deshalb das Erstellen eines Modells, welches den Aktienkurs eines amerikanischen Unternehmens für einen bestimmten Zeitraum voraussagt. Dafür wird unter Verwendung von mehreren Parametern, wie zum Beispiel der Eröffnungs- und Schlusskurs einer Aktie eines Tages, ein Modell mithilfe einer linearen Regression erstellt.

Die Gliederung dieser Arbeit wird sich an dem \ac{CRISP-DM} orientieren. Dieser stellt einen klassischen Ablauf in einem Data-Mining-Projekt dar und findet ebenfalls Verwendung im Bereich der KI. Im Allgemeinen wird dieser Prozess verwendet, um den Lebenszyklus von Daten innerhalb eines Data-Mining-Projekts zu modellieren und ein dafür vorgesehenes Data-Mining-Modell zu entwickeln. Unterteilt wird dieser Prozess in mehrere Phasen, welche das Geschäfts-, das Datenverständnis, die Datenvorbereitung, die Modellierung sowie die Evaluierung und gegebenenfalls das Deployment thematisieren \parencite[vgl.][S. 5ff.]{.Wirth}. 