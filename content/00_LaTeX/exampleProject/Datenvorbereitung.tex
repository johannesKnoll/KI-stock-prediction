\chapter{Datenvorbereitung}\label{chp:datenvorbereitung}
Damit das ML-Modell erfolgreich trainiert werden kann, muss der verwendete Datensatz angepasst und gegebenenfalls bereinigt werden. Da innerhalb des verwendeten Datensatzes keine Duplikate oder fehlerhaften Daten enthalten sind, kann direkt damit begonnen werden, die Daten für das Trainieren des Modells vorzubereiten.

Der Datensatz wurde in einem Dataframe mithilfe der Programmiersprache Python gespeichert. Ein Dataframe ist eine zweidimensionale Datenstruktur, welches Daten in Tabellenform anordnet. Der Datensatz wurde zu Beginn direkt in ein Dataframe überführt. Anschließend wurden für die Berechnungen nicht benötigten Spalten der Datenstruktur entfernt. Dazu zählen beispielsweise das Symbol einer Aktie oder das Datum, welches auf den Handelstag hindeutet. Bei dem trainierten Modell werden jeweils nur die Schlusskurse einer bestimmten Aktie vorhergesagt. Dafür wird der Dataframe zu Beginn auf die Datensätze begrenzt, die dasselbe Symbol besitzen. Um beispielsweise den Aktienkurs von Amazon vorherzusagen, wird das ML-Modell auch nur mit historischen Daten von Amazon trainiert.

Im Anschluss wird der angepasste Datensatz mithilfe der Methode \textit{preprocessing.scale}\footnote{https://scikit-learn.org/stable/modules/generated/sklearn.preprocessing.scale.html} der Bibliothek sckit-learn\footnote{https://scikit-learn.org/stable/index.html} für das Trainieren des Modells standardisiert. Diese Standardisierung wird vorgenommen, damit die Rohdaten, die aus dem Datensatz stammen, so umgewandelt werden, damit sie in dem ML-Modell verarbeitet werden können. Dieser Prozess stellt einen Schritt dar, der in nahezu jedem ML-Projekt vollzogen werden muss.

Danach wird der standardisierte Datensatz in einen großen und einen kleinen Teil aufgeteilt. Der kleine Teil wird im weiteren Verlauf des Projekts verwendet, um einen Aktienkurs vorherzusagen. Die Größe des kleineren Teils beträgt dabei etwa 1\%, also etwa 18, der Gesamtgröße des Datensatzes. Auf dem großen Teil des Datensatzes wird ein Train-Test-Split mit der Methode \textit{train\_test\_split}\footnote{https://scikit-learn.org/stable/modules/generated/sklearn.model\_selection.train\_test\_split.html} durchgeführt. Diese Methode zerteilt den großen Datensatz in ein Trainings- und ein Test-Set. Mit dem Trainings-Set wird das ML-Modell trainiert. Das Test-Set hingegen wird verwendet, um die Qualität der Vorhersagen zu bewerten, die durch das Modell getroffen werden.