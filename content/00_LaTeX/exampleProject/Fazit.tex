\chapter{Fazit und Ausblick}\label{chp:fazit-und-ausblick}
Für das Vorhersagen von dem Schlusskurs einer Aktie wurde in dieser Ausarbeitung ein ML-Modell in Form einer linearen Regression entwickelt, welches mehrere unabhängige Variablen entgegennimmt. Zu diesen Variablen gehören unter anderem der Eröffnungs-, der Schlusskurs oder der höchste Kurs einer Aktie an einem bestimmten Handelstag. Um den Schlusskurs für darauffolgende Tage vorherzusagen, wurde das Modell mit den Variablen trainiert. Im Anschluss werden Vorhersagen für die Schlusskurse getroffen.

Die Evaluierung zeigte auf, dass sich die Vorhersagen der Schlusskurse weitestgehend in dem Bereich der tatsächlichen Schlusskurse befinden. Neben gelegentlichen Ausreißern traten ebenfalls Vorhersagen auf, die nahezu identisch mit dem tatsächlichen Schlusskurs waren. Dennoch wird abschließend festgehalten, dass ein solches Modell nicht verwendet werden sollte, um eine Kaufentscheidung einer Aktie auszusprechen. Da sich der Kurs, der durch das Modell vorhergesagt wird, teilweise gegensätzlich zum tatsächlichen Kurs verhält, kann diese zu hohen Verlusten am Aktienmarkt führen, wenn auf ein solches Modell vertraut wird.

Um die Qualität der Vorhersagen zu verbessern, kann die Anwendung eines anderen Algorithmus in Betracht gezogen werden. Beispielsweise kann mit einem neuronalen Netz gearbeitet werden, mit dem Verbindungen und Abhängigkeiten innerhalb der Trainingsdaten erkannt werden können. 
Eine weitere Verbesserungsmöglichkeit sollte die Berücksichtigung zusätzlicher Faktoren sein, die in das Trainieren des Modells aufgenommen werden, wie zum Beispiel die aktuelle Marktsituation oder Wirtschaftsnachrichten. Jedoch sollte stets festgehalten werden, dass sich der Aktienmarkt sehr schwer gezielt beeinflussen und vorhersagen lässt.
