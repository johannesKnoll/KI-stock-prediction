\chapter{Geschäftsverständnis}\label{chp:geschaeftsverstaendnis}
Die Vorhersage von Aktienkursen ist ein wichtiges Finanzthema, das seit vielen Jahren die Aufmerksamkeit der Forscher auf sich zieht. Dabei wird davon ausgegangen, dass die in der Vergangenheit öffentlich zugänglichen grundlegenden Informationen einen gewissen prädiktiven Einfluss auf die zukünftigen Aktienrenditen haben \parencite[vgl.][S. 1]{.2013}.

Um die Aktienkurse vorherzusagen, existieren grundsätzlich zwei Methoden: die fundamentale und die technische Analyse \parencite[vgl.][S. 26f]{Petrusheva.2016}. Im Bereich der Fundamentalanalyse werden Handelsregeln auf Grundlage der Informationen über Makroökonomie und der Industrie des Unternehmens hergeleitet \parencite[vgl.][S. 1]{.2013}. Diese Art von Analyse geht davon aus, dass der Preis einer Aktie von ihrem inneren Wert und der erwarteten Kapitalrendite abhängt \parencite[vgl.][S. 453]{Tsang.2007}. Somit kann bei dieser Methode beispielsweise anhand der finanziellen Lage, der Liquidität oder der Attraktivität der Branche bewerten, ob eine Aktie Potenzial hat, in der kommenden Zeit zu steigen.

Bei der technischen Analyse hingegen werden Handelsregeln auf Grundlage historischer Daten von Aktienkursen und -volumen entwickelt \parencite[vgl.][S. 1]{.2013}. Diese Methode der Analyse bezieht sich auf verschiedene Methoden, die darauf abzielen, künftige Kursbewegungen anhand von vergangener Aktienkurse und -volumen vorherzusagen. Sie beruht auf der Annahme, dass sich in der Vergangenheit liegende Muster von Aktienkursen wiederholen und dass zukünftige Marktrichtungen anhand von der Untersuchung historischer Kursdaten ermittelt werden können \parencite[vgl.][S. 454]{Tsang.2007}. Bei dieser Methode werden Verfahren wie Trendanalysen, Chartmuster oder technische Indikatoren wie der \ac{MACD} oder Oszillatoren verwendet \parencite[vgl.][S. 293ff]{Mondello.2017}.

Mithilfe dieser Methoden werden Aktienkurse vorausgesagt, um beispielsweise den perfekten Zeitpunkt zu finden, um in die Investition eines Unternehmen einzusteigen oder eine Aktie zu verkaufen, um den Verlust von sinkenden Aktienkursen vorzubeugen. Da diese methodischen Analysen keine Garantie für den Verlauf einer Aktie garantieren, ergibt sich die Überlegung, Aktienkurse mithilfe eines ML-Modells vorherzusagen.